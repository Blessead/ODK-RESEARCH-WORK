\documentclass{article}
\usepackage[margin=0.5in]{geometry}

\begin{document}

\title{ ACONCEPT PAPAER ABOUT  HOSTELS AROUND MAKERERE UNIVERSITY.}

\author{ BY: OBBO PETER}
\date {15/05/2017}


\maketitle


\tableofcontents


\section{INTRODUCTION}\label{sec:into}
Around Makerere University there are numbery of Hostels that set up for the government sponsored students who are non-resident and private student that have pick up interest in any course at the great Makerere University.
	These great hostels are of two categories: -
	\begin{enumerate}
	\item  AFFILITIATED HOSTELS: These are type of hostels that have an official connection with the University and they follow the laws that govern the University.
	\item NON-AFFILIATED HOSTELS: These are the types of hostels that have no connection with the University. Therefore, they exist as stand-alone private entity.
\end{enumerate}

Beholds the choice of these hostels dependents on the someone’s priority since demand is always backed up by ability which has turned into a great challenge into most to the new students who are just joining the University from far areas due to lack of prior knowledge about these hostels thus end up being misled by fellow students who mostly have hidden agendas while making decisions and advising friends thus a challenge and regrets afterwards. 

	

\section{BACKGROUND OF THE PROBLEM}\label{sec:into}

For any student to take-up any type of Hostel either affiliated or non-affiliated involves a process that one has to undergo which includes writing a request letter to the warden or matron or any concerned person of the hostel, in case one is accepted full payments of the accommodation fee proceeds before one is allocated a room.
The request might be bounded in case of no free accommodation space for the applicate which is total disorganization of a student and a burden to look for another hostel for accommodation which requires the same procedures.   


\subsection{PROBLEM STATEMENT}\label{sec:into}
It has been investigated and proved that many new students call them the freshers face a challenge if identifying which hall or hostel to join while pursing his/her studies at Makerere University. 

This has resulted into the problem of being misled in the decision making which can affect the student in her/his entire life while at campus since these halls and hostels are associated with different problems like bed bugs, insecurity, high cost of living among others. Therefore, in so doing this I will greatly help on the decision making while selecting for the hall or hostel of residence of our dear freshers this this research has helped to get the details of almost all halls and hostels affiliated to the University which include accommodation fee, level of security, hygiene, quality of the rooms as well as cost of living around the halls and the hostels. 
 

\section{OBJECTIVES}\label{sec:into}

\subsection{Main Objective}\label{sec:into}
To provide a prior knowledge of the different types of the hostels available around Makerere University and   their services offered the entire new community that is just about to join Makerere University in order to minimize the challenges of being misled by friend and associated problem. 

\subsection{Other Objective}\label{sec:into}
\begin{enumerate}
	\item  To regulate on the misconception that people have about different Hostels.
	\item   To advertise the newly established hostels that are affiliated to the University. 
	\item  To ease identification of hostels around Makerere.
	\item    To increase on sample space when someone is making a decision on which hostel to take up.
	\item   To improve the services offered by these hostels to wide exposure and comparison that will be set up   between different hostels.
\end{enumerate}


\section{SCOPE}\label{sec:into}

This investigation is targeted to newly students joining Makerere and Hostels that are around Makerere University in areas like Kikoni and Wandegeya.
















\end{document}